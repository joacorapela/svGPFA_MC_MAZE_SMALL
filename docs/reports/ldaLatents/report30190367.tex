\documentclass[12pt]{article}

\usepackage{graphicx}
\usepackage{verbatim}
\usepackage[colorlinks=true]{hyperref}
\usepackage{tikz} % for \foreach
\usepackage{amsfonts}
\usepackage{amsmath}
\usepackage[natbibapa]{apacite}

\def\estNumber{30190367}
\def\clustersToPlot{000,001,002,003,004,010,013,014,025}
\def\latentsToPlot{000,001,002,003,004}

\title{svGPFA analysis \texttt{\estNumber} of MC\_MAZE\_SMALL with LDA latents}
\author{Joaquin Rapela}

\begin{document}

\maketitle

\begin{description}
    \item[number of latents:] 15
    \item[number of inducing points:] 20
    \item[number of trials:] 100
    \item[number of clusters:] 142
\end{description}

\section{Behavioral data}

Figure~\ref{fig:behavioralData} plots the behavioral data. In every trial
monkeys reach to one of the nine target locations.

\begin{figure}
    \centering
    \href{https://www.gatsby.ucl.ac.uk/~rapela/svGPFA/reports/mcMazeSmall/figures/00000000_dandisetID000128_epochedEventmove_onset_time_epochedHandPos.html}{\includegraphics[width=5in]{../../../figures/00000000_dandisetID000128_epochedEventmove_onset_time_epochedHandPos.png}}
    \caption{Hand reaching behavior of a monkey. Each trace correspond to a
    different trial. Trials are colored by target location. There are nine
    target locations, three to the upper left quadrants, and two to the other
    quadrants. Click on the figure
    to get its interactive version.}
    \label{fig:behavioralData}
\end{figure}

\section{Data for LDA}

We provided LDA a matrix of data, $X\in\mathbb{R}^{L\times N}$, and a vector of
labels, $\mathbf{y}\in\mathbf{R}^N$. The matrix of data $X$ contains all 15
latents, of all trials, from a 200~ms segment starting at movement onset. A
column of $X$ contains the value of the 15 latents corresponding to one sample
time point of one trial; thus $L=15$. The number of columns of $X$ is the
number of trials times the number of sample points in the 200~ms
segment. The length of $\mathbf{y}$ is the same as the number of columns of
$X$, and the value of $y[i]$ is the target location corresponding to the trial
of the ith column of $X$.

\section{Methods}
\label{sec:methods}

Our implementation of LDA estimated eight (number of target locations minus
one) non-orthogonal directions that maximize a criterion of separability,
$J_1(A)$, of the projected data $Y$, where $Y=AX$, $J_1(A)=\text{Trace}(S_b\
S_w^{-1})$, and $S_b, S_w$ and the between and within group scatter matrices of
Y.
Custom code implementing LDA can be found
\href{https://github.com/joacorapela/svGPFA_MC_MAZE_SMALL/blob/d9b356b179f5e5f27e63437022a55a905b82b099/code/scripts/doPlotLDAtransformedLatents_JR.py}{here}.
Our LDA implementation produced almost identical results as the
\href{https://github.com/joacorapela/svGPFA_MC_MAZE_SMALL/blob/d9b356b179f5e5f27e63437022a55a905b82b099/code/scripts/doPlotLDAtransformedLatents_scikitLearn.py}{scikit-learn}
one.

The rows of matrix $A$ span the LDA space of dimension number of reach targets
minus one. We orthonormalized this space and projected the estimated latents on
the orthonormalized space.

\section{Results}

\subsection{First discriminatory direction separates left from right trials}

Figure~\ref{fig:histProjectionsOnDiscriminatoryDir0} plots histograms of projections of
columns of $X$ onto the first discriminatory directions obtained from LDA.
There are nine histograms, as many of reaching directions. The histogram for
the ith reaching direction contains projections of all columns of $X$
corresponding to trials where the subject reached to the ith direction. The
color of the histogram corresponds to the color of the reaching direction in
Figure~\ref{fig:behavioralData}.
%
The title of the figure shows the eigenvalue corresponding to this
discriminatory direction, which indicates the contribution of this direction to
the optimized separability criterion $J_1(A)$ (Section~\ref{sec:methods}).
$J_1(A)=\text{Trace}(S_b\
S_w^{-1})=\text{eigval}_1+\text{eigval}_2+\cdots+\text{eigval}_{\text{nTargetLocs}-1}$
\citep[][Eq.~10.19]{fukunaga90}.
%
This discriminatory direction separates well right (red and green,
Figure~\ref{fig:behavioralData}) from left (blue and yellow,
Figure~\ref{fig:behavioralData}) trials.

\begin{figure}
    \centering

    \href{https://www.gatsby.ucl.ac.uk/~rapela/svGPFA/reports/mcMazeSmall/figures/30190367_histProjectionsLatentsOntoDiscriminatoryDir00From0.00Duration0.200.html}{\includegraphics[width=5in]{../../../figures/30190367_histProjectionsLatentsOntoDiscriminatoryDir00From0.00Duration0.200.png}}

    \caption{Histograms of projections onto the first LDA direction. Different
    histograms correspond to the different reach target locations show in
    Figure~\ref{fig:behavioralData}.  This LDA direction separates right (red
    and green) from left (blue and yellow) trials. Click on the figure to get
    its interactive version.}

 \label{fig:histProjectionsOnDiscriminatoryDir0}

\end{figure}

Also, latents corresponding to right and left trials are well separated in the
0.0-0.2 time interval when projected onto the first dimension of the
orthonormlised LDA space (see Section~\ref{sec:methods}), as show in
Figure~\ref{fig:latentsProjectedOntoLDA0}.

\begin{figure}
    \centering

    \href{https://www.gatsby.ucl.ac.uk/~rapela/svGPFA/reports/mcMazeSmall/figures/30190367_ldaLatent000.html}{\includegraphics[width=5in]{../../../figures/30190367_ldaLatent000_lightBlueVsLightRed.png}}

    \caption{Latents projected onto the first direction of the orthonormalized
    LDA space. For clarity, the static image on this report displays only
    latents corresponding to trials where the target was on the top-right and
    top left. The interactive version of this image shows all latents. Click on
    the figure to get its interactive version.}

 \label{fig:latentsProjectedOntoLDA0}

\end{figure}

\subsection{Second discriminatory direction separates left from right trials}

Figure~\ref{fig:histProjectionsOnDiscriminatoryDir1} plots histograms of projections of
columns of $X$ onto the second discriminatory directions obtained from LDA.
Note that the eigenvalue corresponding to this discriminatory direction is
almost 80\% smaller than that for the first discriminatory direction,
indicating that this direction contributes almost 80\% less to the
discriminatory criterion $J_1(A)$.
%
This discriminatory direction separates top (red and blue,
Figure~\ref{fig:behavioralData}) from bottom (green and yellow,
Figure~\ref{fig:behavioralData}) trials. However, this separation is weaker
than that with the first discriminative direction
(Figure~\ref{fig:histProjectionsOnDiscriminatoryDir0}).

\begin{figure}
    \centering

    \href{https://www.gatsby.ucl.ac.uk/~rapela/svGPFA/reports/mcMazeSmall/figures/30190367_histProjectionsLatentsOntoDiscriminatoryDir01From0.00Duration0.200.html}{\includegraphics[width=5in]{../../../figures/30190367_histProjectionsLatentsOntoDiscriminatoryDir01From0.00Duration0.200.png}}

    \caption{Histograms of projections onto the second LDA direction. Same
    format as in Figure~\ref{fig:histProjectionsOnDiscriminatoryDir0}.
    This LDA direction separates top (red
    and blue) from bottom (green and yellow) trials. Click on the figure to get
    its interactive version.}

 \label{fig:histProjectionsOnDiscriminatoryDir1}

\end{figure}

Also, latents corresponding to top and bottom trials are well separated in the
0.0-0.2 time interval when projected onto the second dimension of the
orthonormlised LDA space (see Section~\ref{sec:methods}), as show in
Figure~\ref{fig:latentsProjectedOntoLDA1}. However, this separation is weaker
than that in Figure~\ref{fig:latentsProjectedOntoLDA0}. 

\begin{figure}
    \centering

    \href{https://www.gatsby.ucl.ac.uk/~rapela/svGPFA/reports/mcMazeSmall/figures/30190367_ldaLatent001.html}{\includegraphics[width=5in]{../../../figures/30190367_ldaLatent001_darkYellowVsLightBlue.png}}

    \caption{Latents projected onto the second direction of the orthonormalized
    LDA space. For clarity, the static image on this report displays only
    latents corresponding to trials where the target was on the top-left and
    bottom-left. The interactive version of this image shows all latents. Click on
    the figure to get its interactive version.}

 \label{fig:latentsProjectedOntoLDA1}

\end{figure}

\listoffigures

\bibliographystyle{apacite}
\bibliography{signalProcessing}

\end{document}
